\section{Introduction}
Flame propagation in a flow field is an important problem from both theoretical and practical points of view, characterised in general by the interaction of a curved flame with a flow that can involve a wide range of
temporal and spatial scales. This complication is sometimes coupled by the presence of different kinds of inhomogeneity in the combustion mixture itself. These include the spatial non-uniformities in the compositions of the reactants and their temperature, as frequently encountered in non-premixed devices such as that in mixing layers of initially non-premixed reactants. Analysing such problems in real-life applications is a formidable task, but it is instructive to examine such interactions in mixing layers, at least for simple prototypical flows.

\section{Mathematical Model}
Flame propagation in a flow field is an important problem from both theoretical and practical points of view, characterised in general by the interaction of a curved flame with a flow that can involve a wide range of
temporal and spatial scales. This complication is sometimes coupled by the presence of different kinds of inhomogeneity in the combustion mixture itself. These include the spatial non-uniformities in the compositions of the reactants and their temperature, as frequently encountered in non-premixed devices such as that in mixing layers of initially non-premixed reactants. Analysing such problems in real-life applications is a formidable task, but it is instructive to examine such interactions in mixing layers, at least for simple prototypical flows. 

\section{Methodology}

This section contains the methodology used throughout the chapter.

\section{Results}

Describe your  genius  results here :) 