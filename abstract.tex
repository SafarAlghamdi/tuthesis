Describe briefly your thesis in a single page.  

We study in this thesis the influence of prescribed flows on three distinct types of flame, namely flame balls, premixed flames and triple flames. The interaction between flame propagation and fluid flow is examined using asymptotic analyses and numerical simulations of thermo-diffusive models. We consider first the effect of a flow of hot inert gas, either a source or a sink, located at the origin of flame balls. It is shown that the flow gives rise to new kinds of flame balls characterised by having nonzero burning speeds, which we refer to as {\em generalised flame balls}. 

The second part of the thesis is concerned with premixed flame propagation in the presence of parallel and vortical flow fields. Special attention is devoted to examining the effect of high-intensity flow on flame propagation. In this limit, the study identifies several behaviours of the effective flame speed depending on the flow intensity, the flow scale and the Reynolds number. 

Finally, we study the response of triple flames to the presence of parallel flows in the direction of flame propagation. The effect of flow on flame propagation is found to be determined mainly by the scale of the flow compared to the radius of curvature of the flame: large scale flows are shown to have no remarkable influence on the flame structure, whereas flows whose scale is of the same order of magnitude or smaller than the radius of curvature can affect the flame in different ways, such as wrinkling its premixed branches or shifting its leading edge from the stoichiometric line. 