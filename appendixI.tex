We now consider the classical problem with a source of reactive mixture at the origin sustaining
a spherically symmetric flame, mentioned briefly at the end of section~\ref{flamespeed}.
In the limit $\beta \rightarrow \infty$, the reaction is confined to
an infinitely thin reaction sheet, located at $r=r_f$, say. We adopt the near-equidiffusion flame approximation for
 which $l \equiv \beta (\Le - 1)$ is ${\cal{O}}(1)$, supplemented by the assumption that
the temperature at infinity $\theta_{\infty}$ may deviate from unity by
an amount at most of ${\cal{O}}(\beta^{-1})$, and thus may be
written as $\theta_{\infty} = 1+ h_{\infty}/\beta$, which defines $h_{\infty}$. These
assumptions insure that the leading order temperature $\theta^0$ is
unity in the burnt gas ($\theta^{0}(r\geq r_f)=1$), and allow the
problem to be reformulated in terms of   $\theta^0$ and the
(excess) enthalpy $h \equiv \theta^1+y_F^1 \sim \beta (\theta+y_F -1)$.
In terms of $\theta^0$ and $h$ we have to solve
